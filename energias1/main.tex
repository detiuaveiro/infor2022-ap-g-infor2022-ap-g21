\documentclass{report}
\usepackage[T1]{fontenc} % Fontes T1
\usepackage[utf8]{inputenc} % Input UTF8
\usepackage[backend=biber, style=ieee]{biblatex} % para usar bibliografia
\usepackage{csquotes}
\usepackage[portuguese]{babel} %Usar língua portuguesa
\usepackage{blindtext} % Gerar texto automaticamente
\usepackage[printonlyused]{acronym}
\usepackage{hyperref} % para autoref
\usepackage{graphicx}
\usepackage{indentfirst}

\bibliography{bibliografia}


\begin{document}
%%
% Definições
%

\def\titulo{Energias}
\def\data{DATA}
\def\autores{Francisco Franco, Tomás Oliveira}
\def\autorescontactos{(113275) franco.f04@ua.pt, (113939) tomas.esteves.oliveira@ua.pt}
\def\versao{VERSAO 1}
\def\departamento{Dept. de Eletrónica, Telecomunicações e Informática}
\def\empresa{Universidade de Aveiro}
\def\logotipo{ua.pdf}
%
%%%%%% CAPA %%%%%%
%
\begin{titlepage}

\begin{center}
%
\vspace*{50mm}
%
{\Huge \ Energias}\\ 
%
\vspace{10mm}
%
{\Large \empresa}\\
%
\vspace{10mm}
%
{\LARGE \autores}\\ 
%
\vspace{30mm}
%
\begin{figure}[h]
\center
\includegraphics{\logotipo}
\end{figure}
%
\vspace{30mm}
\end{center}
%
\begin{flushright}
\versao
\end{flushright}
\end{titlepage}

%%  Página de Título %%
\title{%
{\Huge\textbf{\titulo}}\\
{\Large \departamento\\ \empresa}
}
%
\author{%
    \autores \\
    \autorescontactos
}
%
\date{\today}
%
\maketitle

\pagenumbering{roman}

%%%%%% RESUMO %%%%%%
\begin{abstract}
Resumo de 200-300 palavras.
\end{abstract}

%%%%%% Agradecimentos %%%%%%
% Segundo glisc deveria aparecer após conclusão...
\renewcommand{\abstractname}{Agradecimentos}
\begin{abstract}
Gostariamos de agradecer, de uma forma breve, ao nosso professor desta unidade curricular António Manuel Adrego da Rocha pela maneira cativante como ensina os alunos durante as aulas de modo a obtermos as melhores técnicas/ferramentas para executar os trabalhos que nos forem propostos.
“Ser professor não é só uma questão de possuir um corpo de
conhecimentos e capacidade de controlo da aula. Isso
poderia fazer-se com um computador e um bastão. Para ser
professor é preciso, igualmente, ter capacidade de
estabelecer relações humanas com as pessoas a quem se
ensina. Aprender é um processo social humano e árduo, o
mesmo se pode dizer de ensinar. Ensinar implica,
simultaneamente, emoções e razão pura". 
\end{abstract}


\tableofcontents
% \listoftables     % descomentar se necessário
% \listoffigures    % descomentar se necessário


%%%%%%%%%%%%%%%%%%%%%%%%%%%%%%%
\clearpage
\pagenumbering{arabic}

%%%%%%%%%%%%%%%%%%%%%%%%%%%%%%%%
\chapter{Introdução}
\label{chap.introducao}
\Large
No âmbito do relatório técnico da unidade curricular de IEI (Introdução à Engenharia informática) foi-nos atribuída a exploração de um tema à nossa escolha e decidimos optar pelas Energias. Neste trabalho iremos explicar, de uma maneira clara e objetiva o que são estas energias, referir as vantagens e algumas desvantagens que estas trazem para a sociedade. Iremos também apresentar os principais tipos de energias bem como quando estas começaram a ser aplicadas. Consideramos que falar sobre este tema é indispensável, especialmente nos dias de hoje devido à guerra, pois estas energias renováveis são recursos inesgotáveis e evitam que Portugal seja dependente de outros países devido à importação de combustíveis fósseis, como é o caso do carvão e/ou o gás natural.


\chapter{Metodologia}
Neste projeto começamos por decidir qual o tema iriamos abordar acabando assim por optar pelas Energias devido à grande consideração e relevância que ambos atribuímos a esta temática. O trabalho foi feito maioritariamente de forma distanciada, utilizando métodos que nos foram introduzidos na unidade curricular de IEI, em que a cada elemento do grupo foram atribuídas tarefas e foram executadas com sucesso.

\label{chap.metodologia}

\chapter{Energias Renováveis}
\section{O que são as Energias Renováveis}
Num primeiro momento desta apresentação iremos explicitar o que são energias renováveis. O termo renovável demonstra que algo, nomeadamente uma energia, está disponível na natureza e tem a habilidade de se regenerar continuamente, em quantidades praticamente inesgotáveis. Por assim dizer estas energias provêm de recursos naturais que não têm limites daí a importantíssima razão para as utilizarmos.

\section{Principais tipos de Energias Renováveis}
Diferentes tipos de Energias Renováveis correspondem ás maneiras pelas quais podemos, devido aos avanços tecnológicos, produzir energia através de distintas fontes naturais. Neste trabalho iremos abordar apenas as principais métodos como por exemplo pelo meio solar, eólico e finalmente hidráulico.


\subsection{Energia Solar}
A Energia Solar 
\subsection{Energia Eólica}
\subsection{Energia Hidráulica}
\section{Quando é que começaram a ser aplicadas}
\section{Qual é a maior fonte de energia na natureza}
\section{Fatores a considerar:}
\subsection{Vantagens}
\subsection{Desvantagens}
\chapter{Energias não renováveis}
\section{Principais energias não renováveis}
\section{Desvantagens}


\label{chap.resultados}
Descreve os resultados obtidos.

\chapter{Confrontação das energias}
\label{chap.analise}
Analisa os resultados.

\chapter{Conclusões}
\label{chap.conclusao}
Apresenta conclusões.

\chapter*{Contribuições dos autores}
Resumir aqui o que cada autor fez no trabalho.
Usar abreviaturas para identificar os autores,
por exemplo AS para António Silva.

\vspace{10pt}
\textbf{Indicar a percentagem de contribuição de cada autor.}\\

\autores : xx\%, yy\%\\

%%%%%%%%%%%%%%%%%%%%%%%%%%%%%%%%%
\chapter*{Acrónimos}
\begin{acronym}
\acro{ua}[UA]{Universidade de Aveiro}
\acro{leci}[LECI]{Licenciatura em Engenharia de Computadores e Informática}
\acro{ee}[EE]{Energias}
\acro{Francisco}[FF]{Francisco Franco}
\acro{Tomás}[TO]{Tomás Oliveira}
\end{acronym}


%%%%%%%%%%%%%%%%%%%%%%%%%%%%%%%%%
\printbibliography

\end{document}