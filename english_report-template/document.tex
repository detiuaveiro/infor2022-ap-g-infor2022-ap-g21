\documentclass{report}
\usepackage[T1]{fontenc} % Fonts T1
\usepackage[utf8]{inputenc} % Input UTF8
\usepackage[backend=biber, style=ieee]{biblatex} % for using bibliography
\usepackage{csquotes}
\usepackage[english]{babel} %Using english llanguage
\usepackage{blindtext} % Generating text automatically
\usepackage[printonlyused]{acronym}
\usepackage{hyperref} % for autoref
\usepackage{graphicx}
\usepackage{indentfirst}

\bibliography{bibliography}


\begin{document}
%%
% Definitions
%
\def\titledoc{REPORT TITLE}
\def\datedoc{DATE}
\def\authors{Author1, Author2}
\def\authorscontacts{(nmec1) author1@ua.pt, (nmec2) author2@ua.pt}
\def\version{VERSION}
\def\department{Dept. de Eletrónica, Telecomunicações e Informática}
\def\company{Universidade de Aveiro}
\def\logo{ua.pdf}
%
%%%%%% COVER %%%%%%
%
\begin{titlepage}

\begin{center}
%
\vspace*{50mm}
%
{\Huge \titledoc}\\ 
%
\vspace{10mm}
%
{\Large \company}\\
%
\vspace{10mm}
%
{\LARGE \authors}\\ 
%
\vspace{30mm}
%
\begin{figure}[h]
\center
\includegraphics{\logo}
\end{figure}
%
\vspace{30mm}
\end{center}
%
\begin{flushright}
\version
\end{flushright}
\end{titlepage}

%%  Title Page %%
\title{%
{\Huge\textbf{\titledoc}}\\
{\Large \department\\ \company}
}
%
\author{%
    \authors \\
    \authorscontacts
}
%
\date{\today}
%
\maketitle

\pagenumbering{roman}

%%%%%% SUMMARY %%%%%%
\begin{abstract}
Summary of 200-300 words.
\end{abstract}

%%%%%% Acknowledgments %%%%%%
% According to glisc should appear after conclusions ...
\renewcommand{\abstractname}{Acknowledgments}
\begin{abstract}
Any acknowledgments.
Comment block if no acknowledgments are due.
\end{abstract}


\tableofcontents
% \listoftables     % descomentar se necessário
% \listoffigures    % descomentar se necessário


%%%%%%%%%%%%%%%%%%%%%%%%%%%%%%%
\clearpage
\pagenumbering{arabic}

%%%%%%%%%%%%%%%%%%%%%%%%%%%%%%%%
\chapter{Introduction}
\label{chap.intro}

It introduces the theme, presents the motivation and finally the structure.

This document is divided into four chapters.
After this introduction,
in \autoref{chap.methods} the methodology followed is presented,
in \autoref{chap.results} the results obtained are displayed,
these being discussed in \autoref{chap.analysis}.
Finally, in \autoref{chap.conclusions}
the conclusions of the work.

\chapter{Methodology}
\label{chap.methods}
Describes the methods used to obtain results.

In this report skeleton, we take advantage of this chapter to exemplify
how to use some elements {\LaTeX}.

\section{Exempls}

\subsection{Using acronyms}
This is the first invocation of the acronym \ac{ua}.
And this is the second one \ac{ua}.

Another reference to \ac{leci}.

\subsection{References}
Information regarding the formal structure of a report can be obtained
on the page of \ac{glisc}\cite{glisc}.

\chapter{Results}
\label{chap.results}
It describes the results obtained.

\chapter{Analysis}
\label{chap.analysis}
Analize the results.

\chapter{Conclusions}
\label{chap.conclusions}
Presenting conclusions.

\chapter*{Authors contributions}
Sumarize here what each author did on the job.
Use abbreviations to identify authors,
for example AS for António Silva.

\vspace{10pt}
\textbf{Indicate the percentage of contribution of each author.}\\

\authors : xx\%, yy\%\\

%%%%%%%%%%%%%%%%%%%%%%%%%%%%%%%%%
\chapter*{Acronyms}
\begin{acronym}
\acro{ua}[UA]{Universidade de Aveiro}
\acro{leci}[LECI]{Licenciatura em Engenharia de Computadores e Informática}
\acro{glisc}[GLISC]{Grey Literature International Steering Committee}
\end{acronym}


%%%%%%%%%%%%%%%%%%%%%%%%%%%%%%%%%
\printbibliography

\end{document}
